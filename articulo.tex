\documentclass[12pt]{article}
\usepackage[utf8]{inputenc}
\usepackage[spanish]{babel}
\usepackage{amsmath, amssymb, amsthm}
\usepackage{graphicx}
\usepackage{booktabs}
\usepackage{algorithm}
\usepackage{algpseudocode}
\usepackage{hyperref}
\usepackage{multirow}

\title{Algoritmo Wagner-Whitin: Modelo Dinámico de Dimensionamiento de Lotes}
\author{Martin Rojas Medrano}
\date{\today}

\newtheorem{theorem}{Teorema}
\newtheorem{lemma}{Lema}
\newtheorem{definition}{Definición}
\newtheorem{corollary}{Corolario}
\newtheorem{proposition}{Proposición}
\newtheorem{remark}{Observación}
\newtheorem{example}{Ejemplo}

\begin{document}

\maketitle

\begin{abstract}
Este artículo presenta una exposición exhaustiva del algoritmo Wagner-Whitin para la solución óptima del problema de dimensionamiento de lotes con demanda dinámica y determinística. Partiendo del artículo fundacional de Wagner y Whitin (1958), se desarrolla el marco teórico del modelo, se demuestran los teoremas fundamentales que sustentan el algoritmo, se describe el procedimiento computacional y se ilustra con ejemplos numéricos. El análisis revela la conexión entre la programación dinámica y la gestión de inventarios, destacando el concepto de horizonte de planificación como herramienta para reducir la complejidad computacional. Se incluyen definiciones formales de los conceptos clave y se discuten extensiones modernas del modelo.
\end{abstract}

\section{Introducción}
El problema de dimensionamiento de lotes (\textit{lot sizing}) constituye uno de los problemas fundamentales en la gestión de inventarios. Mientras que el modelo económico clásico (\textit{Economic Order Quantity - EOQ}) supone una demanda constante a través del tiempo, en entornos reales la demanda frecuentemente exhibe variaciones significativas entre periodos.

El algoritmo Wagner-Whitin, introducido en 1958 por Harvey M. Wagner y Thomson M. Whitin, representa la primera solución óptima para el problema de dimensionamiento de lotes con demanda dinámica y determinística. Este algoritmo emplea programación dinámica para determinar cuándo y cuánto ordenar, minimizando los costos totales de \textit{setup} y mantenimiento de inventario a lo largo de un horizonte finito de planificación.

La relevancia del algoritmo persiste hasta la actualidad, sirviendo como base para numerosas extensiones que incorporan restricciones de capacidad, múltiples productos y condiciones de incertidumbre. Este trabajo busca presentar una exposición integral del algoritmo, desde sus fundamentos teóricos hasta su implementación práctica.

\section{Marco Teórico}

\subsection{Fundamentos de Teoría de Inventarios}

\begin{definition}[Sistema de Inventarios]
Un sistema de inventarios consiste en los procesos de adquisición, almacenamiento y distribución de materiales, con el objetivo de garantizar la disponibilidad de productos mientras se minimizan los costos asociados.
\end{definition}

\begin{definition}[Costos Relevantes]
En un sistema de inventarios, se consideran típicamente los siguientes costos:
\begin{itemize}
    \item \textbf{Costo de ordenar/setup} ($s_t$): Costo fijo incurrido al realizar un pedido o iniciar una producción
    \item \textbf{Costo de mantener inventario} ($h_t$): Costo por unidad almacenada por periodo
    \item \textbf{Costo de \textit{shortage}} ($p_t$): Costo por unidad de demanda insatisfecha (no considerado en el modelo Wagner-Whitin clásico)
\end{itemize}
\end{definition}

\begin{definition}[Modelo EOQ Clásico]
El modelo de Cantidad Económica de Pedido supone demanda constante $D$, costo de ordenar $S$, costo de mantener $H$ por unidad por tiempo. El problema que resuelve el modelo EOQ consiste en encontrar la cantidad de pedido $Q$ que minimice la función de costo total por unidad de tiempo:

\[
TC(Q) = \frac{DS}{Q} + \frac{HQ}{2}
\]

donde el primer término representa el costo total de ordenar y el segundo término el costo total de mantener inventario.

La solución óptima a este problema está dada por:

\[
Q^* = \sqrt{\frac{2DS}{H}}
\tag{1}
\]

que produce un costo total mínimo por unidad de tiempo:

\[
TC^* = \sqrt{2DSH}
\tag{2}
\]

Sin embargo, el modelo EOQ resulta inadecuado cuando la demanda varía significativamente entre periodos, lo que motiva el desarrollo de modelos dinámicos como el de Wagner-Whitin.
\end{definition}

\subsection{Programación Dinámica}

El algoritmo Wagner-Whitin se fundamenta en la programación dinámica, técnica desarrollada por Richard Bellman.

\begin{definition}[Principio de Optimalidad de Bellman]
``Una política óptima tiene la propiedad de que, cualesquiera que sean el estado inicial y la decisión inicial, las decisiones restantes deben constituir una política óptima con respecto al estado resultante de la primera decisión.''
\end{definition}

\begin{definition}[Problema de Optimización en Etapas]
Un problema de optimización en etapas (o de programación dinámica) consiste en encontrar una secuencia de decisiones óptimas a lo largo de un horizonte finito de $T$ etapas. Formalmente, se define mediante los siguientes elementos:

\begin{itemize}
    \item \textbf{Horizonte temporal}: $t = 1, 2, \dots, T$ momentos discretos en los que se toman decisiones. Cada etapa $t$ corresponde a un subproblema.
    \item \textbf{Estado del sistema} $s_t \in S_t$: Describe la situación del sistema al inicio de la etapa $t$. El estado $s_t$ resume toda la información relevante para tomar decisiones futuras. $S_t$ es el espacio de estados en la etapa $t$.
    \item \textbf{Decisión o control}: $x_t \in X_t(s_t)$, donde $X_t(s_t)$ es el conjunto de decisiones factibles en el estado $s_t$
    \item \textbf{Función de transición}: $s_{t+1} = g_t(s_t, x_t)$, determina cómo evoluciona el sistema al tomar la decisión $x_t$ en el estado $s_t$
    \item \textbf{Costo inmediato}: $c_t(s_t, x_t)$, costo incurrido en la etapa $t$
    \item \textbf{Costo terminal}: $V_{T+1}(s_{T+1})$, costo asociado al estado final (usualmente 0)
\end{itemize}

El objetivo es encontrar una política $\pi = (x_1, x_2, \dots, x_T)$ que minimice el costo total esperado:

\[
J_\pi(s_1) = \sum_{t=1}^{T} c_t(s_t, x_t) + V_{T+1}(s_{T+1})
\]
\end{definition}

\begin{definition}[Ecuación de Bellman para Gestión de Inventarios]
En el contexto del problema de dimensionamiento de lotes, la ecuación de Bellman describe la política óptima de decisión periodo a periodo. Para cada instante $t = 1, 2, \ldots, N$:

\begin{equation}
f_t(I) = \min_{x_t \geq 0} \left\{ 
\begin{array}{ll}
h_{t-1} \cdot I + \delta(x_t) \cdot s_t + f_{t+1}(I + x_t - d_t) & \text{si } I + x_t \geq d_t \\
\infty & \text{en otro caso}
\end{array}
\right.
\tag{3}
\end{equation}

donde:
\begin{itemize}
    \item $f_t(I)$: \textbf{Función de Bellman} en el periodo $t$ con inventario inicial $I$, que representa el costo mínimo acumulado desde $t$ hasta el horizonte $N$
    \item $t$: \textbf{Índice temporal} que varía en $1 \leq t \leq N$ (periodos discretos)
    \item $I$: \textbf{Nivel de inventario} al inicio del periodo $t$
    \item $x_t$: \textbf{Cantidad a ordenar} en el periodo $t$ (variable de decisión)
    \item $h_{t-1}$: \textbf{Costo unitario de mantenimiento} del periodo $t-1$ a $t$
    \item $\delta(x_t)$: \textbf{Función indicadora} de orden ($\delta(x_t) = 1$ si $x_t > 0$, $0$ en otro caso)
    \item $s_t$: \textbf{Costo fijo de ordenar} en el periodo $t$
    \item $d_t$: \textbf{Demanda} en el periodo $t$
\end{itemize}

La \textbf{condición terminal} completa la definición recursiva:
\begin{equation}
f_{N+1}(I) = \begin{cases} 
0 & \text{si } I = 0 \\
\infty & \text{si } I > 0
\end{cases}
\quad \text{para todo } I \geq 0
\tag{4}
\end{equation}
\end{definition}

\subsection{Optimización Convexa y Estructuras Especiales}

Aunque el problema de Wagner-Whitin incluye costos fijos (no convexos), la estructura especial del problema permite encontrar soluciones óptimas globales.

\begin{definition}[Función Convexa]
Una función $f: \mathbb{R}^n \rightarrow \mathbb{R}$ es convexa si para todo $x, y \in \mathbb{R}^n$ y $\lambda \in [0,1]$:
\begin{equation}
f(\lambda x + (1-\lambda)y) \leq \lambda f(x) + (1-\lambda)f(y)
\tag{5}
\end{equation}
\end{definition}

\begin{definition}[Función Cóncava]
Una función $f: \mathbb{R}^n \rightarrow \mathbb{R}$ es cóncava si $-f$ es convexa.
\end{definition}

La formulación de Wagner-Whitin, aunque no estrictamente convexa debido a la presencia de costos fijos discontinuos, exhibe propiedades estructurales que garantizan la optimalidad global de la solución encontrada mediante programación dinámica. Estas propiedades fundamentales son:

\begin{enumerate}
    \item \textbf{Propiedad de Planificación de Cero Inventarios (Zero-Inventory Ordering)}: En una política óptima, la ordenación ocurre únicamente cuando el nivel de inventario es cero. Formalmente, existe una solución óptima que satisface:
    \begin{equation}
        I_{t-1} \cdot x_t = 0 \quad \text{para todo } t = 1, \ldots, N
        \tag{6}
    \end{equation}
    donde $I_{t-1}$ es el inventario inicial del periodo $t$ y $x_t$ es la cantidad ordenada.
    
    \item \textbf{Estructura de Subproblemas Independientes}: El problema de $N$ periodos se descompone en subproblemas más simples mediante el principio de optimalidad de Bellman. La función de valor $f_t(I)$ satisface:
    \begin{equation}
        f_t(I) = \min_{x_t \geq 0} \left\{ C_t(I, x_t) + f_{t+1}(I + x_t - d_t) \right\}
        \tag{7}
    \end{equation}
    donde $C_t(I, x_t)$ representa el costo inmediato en el periodo $t$.
    
    \item \textbf{Propiedad de Monotonicidad de la Función de Valor}: La función de valor óptimo $f_t(I)$ es no decreciente en el inventario $I$ y no creciente en el tiempo $t$, lo que permite podar soluciones subóptimas durante la búsqueda.
    
    \item \textbf{Existencia de Puntos de Regeneración (Regeneration Points)}: Los periodos en los que el inventario se agota (puntos de regeneración) dividen el problema en subintervalos independientes, reduciendo la complejidad computacional de $O(2^N)$ a $O(N^2)$.
\end{enumerate}

Estas propiedades garantizan que el algoritmo de Wagner-Whitin, basado en programación dinámica, encuentre la solución óptima global a pesar de la no convexidad introducida por los costos fijos. La optimalidad se demuestra mediante inducción hacia atrás sobre el horizonte de planificación, donde en cada etapa $t$ se garantiza que $f_t(I)$ representa el costo mínimo acumulado desde $t$ hasta el final.

\subsection{Planificación Horizontal}

Un concepto fundamental introducido por Wagner y Whitin es el de \textit{horizonte de planificación}, que permite dividir el problema en subproblemas independientes.

\begin{definition}[Horizonte de Planificación]
Un periodo $t$ constituye un horizonte de planificación si la solución óptima para los primeros $t$ periodos es independiente de los datos de demanda y costos más allá del periodo $t$.

\textbf{Ejemplo:} Considere un problema con 4 periodos y demandas $d = [50, 60, 70, 80]$. Si al resolver el problema para 3 periodos encontramos que la última orden en la solución óptima ocurre en el periodo 3 (cubriendo $d_3$), entonces el periodo 3 es un horizonte de planificación. Las decisiones óptimas para los periodos 1-3 permanecen invariantes independientemente de la demanda del periodo 4.
\end{definition}

\begin{definition}[Propiedad de Extensibilidad]
Un problema de planificación posee la propiedad de extensibilidad si la solución óptima para un horizonte $T$ puede extenderse a un horizonte $T+1$ sin modificar las decisiones anteriores.

\textbf{Ejemplo:} Supongamos que para $T=3$ periodos la política óptima es:
\begin{itemize}
    \item Ordenar 110 unidades en $t=1$ (cubre $d_1=50$ y $d_2=60$)
    \item Ordenar 70 unidades en $t=3$ (cubre $d_3=70$)
\end{itemize}
Si al agregar un cuarto periodo con $d_4=80$, la política óptima mantiene las mismas decisiones en $t=1$ y $t=3$, y simplemente añade una orden en $t=4$ para $d_4$, entonces el problema tiene la propiedad de extensibilidad.
\end{definition}

Estos conceptos tienen implicaciones prácticas significativas, pues permiten tomar decisiones óptimas sin conocer el futuro completo. En el algoritmo de Wagner-Whitin, la identificación de horizontes de planificación permite:

\begin{example}[Aplicación en Wagner-Whitin]
Considere un escenario con:
\begin{align*}
d &= [20, 50, 10, 30, 40] \\
s &= [100, 100, 100, 100, 100] \quad \text{(costos fijos)} \\
h &= [1, 1, 1, 1, 1] \quad \text{(costos de mantenimiento)}
\end{align*}

El algoritmo puede determinar que el periodo 2 es un horizonte de planificación, lo que significa que las decisiones óptimas para los periodos 1-2 no dependen de los periodos 3-5. Esto reduce el espacio de búsqueda y simplifica el cómputo.
\end{example}

\section{Modelo Matemático y Preliminares}

\subsection{Definición Formal del Problema}
Considérese un horizonte de planificación discreto de $N$ periodos. En cada periodo $t = 1, 2, \ldots, N$, se definen los siguientes elementos:

\textbf{Parámetros del problema (datos conocidos):}
\begin{itemize}
    \item $d_t$: demanda en el periodo $t$ (unidades)
    \item $s_t$: costo fijo de \textit{setup} (ordenar o producir) en el periodo $t$ (\$)
    \item $h_t$: costo unitario de mantener inventario del periodo $t$ al $t+1$ (\$/unidad/periodo)
\end{itemize}

\textbf{Variables de decisión:}
\begin{itemize}
    \item $x_t$: cantidad a ordenar (o producir) en el periodo $t$ (unidades)
    \item $I_t$: nivel de inventario al final del periodo $t$ (unidades)
\end{itemize}

\textbf{Función auxiliar:}
\begin{itemize}
    \item $\delta(x_t)$: función indicadora que toma valor 1 si $x_t > 0$ y 0 en caso contrario
\end{itemize}

\begin{definition}[Problema de Dimensionamiento de Lotes con Demanda Dinámica]
El problema consiste en encontrar una política de pedidos $\{x_t\}_{t=1}^N$ que minimice:
\begin{equation}
\min \sum_{t=1}^N \left[ \delta(x_t) s_t + h_t I_t \right]
\tag{8}
\end{equation}
sujeto a las restricciones de balance de inventario y no negatividad:
\begin{align}
I_t &= I_{t-1} + x_t - d_t, & \forall t = 1, \ldots, N \label{eq:balance} \\
I_t &\geq 0, & \forall t = 1, \ldots, N \label{eq:no_negatividad_I} \\
x_t &\geq 0, & \forall t = 1, \ldots, N \label{eq:no_negatividad_x} \\
I_0 &= 0 & \text{(inventario inicial cero)} \label{eq:inicial}
\end{align}
donde $\delta(x_t)$ es la función indicadora que toma valor 1 si $x_t > 0$ y 0 en caso contrario.
\end{definition}


\begin{proposition}[Análisis de Factibilidad y Existencia]
El problema de dimensionamiento de lotes con demanda dinámica satisface:

\textbf{(a) Factibilidad:} El problema siempre es factible. Dada cualquier secuencia de demandas no negativas $\{d_t\}_{t=1}^N$, existe al menos una política factible. En particular, la política de \textit{lote por lote} (ordenar exactamente $x_t = d_t$ en cada periodo $t$) es siempre factible.

\textbf{(b) Existencia de solución óptima:} Siempre existe una solución óptima finita. Esto se garantiza porque:
\begin{itemize}
    \item El conjunto factible es no vacío y cerrado
    \item Los costos $s_t > 0$ y $h_t \geq 0$ son acotados
    \item La función objetivo está acotada inferiormente por 0
    \item Existe un número finito de políticas óptimas candidatas debido al \textbf{Teorema de Wagner-Whitin} (ordenar solo cuando el inventario es cero)
\end{itemize}

\begin{proof}[Esbozo de demostración de existencia]
Sea $\mathcal{F}$ el conjunto de políticas factibles. Como la política lote-por-lote pertenece a $\mathcal{F}$, tenemos $\mathcal{F} \neq \emptyset$. Además, $\mathcal{F}$ es cerrado pues está definido por igualdades y desigualdades lineales. La función objetivo $f(x,I) = \sum_{t=1}^N [\delta(x_t)s_t + h_t I_t]$ es no negativa y propia. Por el teorema de Weierstrass, al estar minimizando una función semicontinua inferiormente sobre un conjunto cerrado no vacío, existe al menos un minimizador global.
\end{proof}

\textbf{(c) Caracterización de optimalidad:} Por el Teorema Fundamental de Wagner-Whitin, existe una solución óptima que satisface:
\begin{equation}
I_{t-1} \cdot x_t = 0 \quad \forall t = 1, \ldots, N
\end{equation}
es decir, nunca se ordena cuando hay inventario positivo (\textit{zero-inventory ordering property}).
\end{proposition}

\begin{remark}
La formulación con función indicadora $\delta(x_t)$ hace que el problema sea de \textbf{optimización no lineal con costos fijos}. Sin embargo, la estructura especial del problema garantiza que podemos encontrar la solución óptima global en tiempo polinomial mediante programación dinámica.
\end{remark}

\begin{remark}
El supuesto de inventario inicial cero ($I_0 = 0$) no es restrictivo. Si $I_0 > 0$, el problema puede transformarse equivalentemente considerando la demanda neta $d_1' = \max(0, d_1 - I_0)$ y procediendo con el algoritmo estándar.
\end{remark}

\subsection{Formulación como Programación Entera Mixta}

El problema de dimensionamiento de lotes puede formularse como un programa lineal entero mixto (MILP) mediante la introducción de variables binarias que linealizan la función indicadora $\delta(x_t)$.

\begin{definition}[Formulación MILP del Problema Wagner-Whitin]
Definimos las siguientes \textbf{variables de decisión}:
\begin{itemize}
    \item $x_t \geq 0$: cantidad a ordenar en el periodo $t$ (variable continua)
    \item $I_t \geq 0$: nivel de inventario al final del periodo $t$ (variable continua)
    \item $y_t \in \{0,1\}$: variable binaria que indica si se realiza un pedido en el periodo $t$
\end{itemize}

El problema de optimización consiste en:
\begin{align}
\min_{x, I, y} \quad & \sum_{t=1}^N \left[ s_t y_t + h_t I_t \right] \tag{9} \\
\text{sujeto a:} \quad & I_t = I_{t-1} + x_t - d_t, \quad \forall t = 1, \ldots, N \tag{10} \\
& x_t \leq M_t y_t, \quad \forall t = 1, \ldots, N \tag{11} \\
& I_t \geq 0, \quad x_t \geq 0, \quad \forall t = 1, \ldots, N \tag{12} \\
& y_t \in \{0,1\}, \quad \forall t = 1, \ldots, N \tag{13} \\
& I_0 = 0 \tag{14}
\end{align}
donde $M_t = \sum_{i=t}^N d_i$ es una cota superior para $x_t$ en cualquier solución óptima.
\end{definition}

\begin{proposition}[Equivalencia entre Formulaciones]
Las formulaciones original (6)-(10) y MILP (9)-(14) son equivalentes, meaning:
\begin{enumerate}
    \item Tienen el \textbf{mismo valor óptimo} de la función objetivo
    \item Existe una \textbf{correspondencia biyectiva} entre soluciones óptimas
    \item En toda solución óptima del MILP se cumple $y_t = \delta(x_t) = \mathbb{1}_{\{x_t > 0\}}$
\end{enumerate}
\end{proposition}

\begin{proof}[Demostración de Equivalencia]
La equivalencia se establece mediante dos transformaciones:

\textbf{Transformación del problema original al MILP:}
Dada una solución factible $(x_t, I_t)$ del problema original, definimos:
\[
y_t = \delta(x_t) = \begin{cases} 
1 & \text{si } x_t > 0 \\
0 & \text{si } x_t = 0 
\end{cases}
\]
Esta tripleta $(x_t, I_t, y_t)$ es factible para el MILP porque:
\begin{itemize}
    \item Las restricciones (10), (12) y (14) se mantienen directamente
    \item La restricción (11) se satisface: si $x_t > 0$ entonces $y_t = 1$ y $x_t \leq M_t$; si $x_t = 0$ entonces $y_t = 0$ y $x_t = 0 \leq 0$
    \item El valor objetivo coincide: $\delta(x_t)s_t = y_t s_t$
\end{itemize}

\textbf{Transformación del MILP al problema original:}
Dada una solución factible $(x_t, I_t, y_t)$ del MILP, la pareja $(x_t, I_t)$ es factible para el problema original porque:
\begin{itemize}
    \item Las restricciones (7)-(10) se satisfacen directamente
    \item La función objetivo del problema original evaluada en $(x_t, I_t)$ es:
    \[
    \sum_{t=1}^N [\delta(x_t)s_t + h_t I_t] \leq \sum_{t=1}^N [y_t s_t + h_t I_t]
    \]
    pues $\delta(x_t) \leq y_t$ (si $y_t = 0$ entonces $x_t = 0$ por (11), luego $\delta(x_t) = 0$; si $y_t = 1$ entonces $\delta(x_t) \leq 1 = y_t$)
\end{itemize}

\textbf{Optimalidad y correspondencia:}
En una solución óptima del MILP, necesariamente se cumple $\delta(x_t) = y_t$ para todo $t$, pues de lo contrario existiría algún $t$ con $y_t = 1$ pero $x_t = 0$, y podríamos hacer $y_t = 0$ reduciendo el costo sin violar factibilidad. Por lo tanto, en el óptimo las funciones objetivo coinciden exactamente y las soluciones son equivalentes.
\end{proof}

\begin{remark}[Mecanismo de la Formulación MILP]
La clave de la equivalencia reside en la restricción (11) $x_t \leq M_t y_t$ que:
\begin{itemize}
    \item \textbf{Forza la relación lógica}: Si $x_t > 0$ entonces $y_t = 1$ (pues si $y_t = 0$, la restricción impondría $x_t = 0$)
    \item \textbf{Permite la minimización}: El algoritmo de optimización puede "elegir" $y_t = 0$ cuando $x_t = 0$ para ahorrar el costo $s_t$
    \item \textbf{Mantiene factibilidad}: No excluye ninguna solución óptima del problema original
\end{itemize}
\end{remark}

\begin{remark}[Optimalidad de la Cota $M_t$]
La elección $M_t = \sum_{i=t}^N d_i$ es \textbf{óptima} porque:
\begin{itemize}
    \item Es \textbf{suficiente}: En cualquier solución óptima, $x_t \leq \sum_{i=t}^N d_i$ debido a que ordenar más que la demanda total restante sería subóptimo
    \item Es \textbf{ajustada}: Existen soluciones óptimas donde $x_t = \sum_{i=t}^N d_i$ (política de ordenar todo al inicio)
    \item Evita \textbf{relajaciones débiles}: Una $M_t$ más grande debilitaría la relajación lineal del problema
\end{itemize}
\end{remark}

\begin{example}[Ilustración de la Transformación]
Considere una solución óptima del problema original para $N=3$ con:
\[
x_1 = 50, x_2 = 0, x_3 = 40; \quad I_1 = 30, I_2 = 0, I_3 = 0
\]
con demandas $d = [20, 30, 40]$.

La solución equivalente en el MILP es:
\[
x_1 = 50, x_2 = 0, x_3 = 40; \quad I_1 = 30, I_2 = 0, I_3 = 0; \quad y_1 = 1, y_2 = 0, y_3 = 1
\]

Ambas tienen el mismo costo: $s_1 + s_3 + h_1 \cdot 30$.
\end{example}

\begin{remark}[Complejidad Computacional]
Aunque equivalentes matemáticamente, las formulaciones difieren computacionalmente:
\begin{itemize}
    \item \textbf{Formulación original}: Problema no lineal (NP-duro en general)
    \item \textbf{Formulación MILP}: Problema lineal entero mixto, pero con $N$ variables binarias
    \item \textbf{Algoritmo Wagner-Whitin}: Explota la estructura específica para resolver en $O(N^2)$
\end{itemize}
Para problemas grandes, el algoritmo específico sigue siendo más eficiente que resolver el MILP con métodos generales.
\end{remark}
\subsection{Formulación como Programación Dinámica}

El problema de dimensionamiento de lotes admite una formulación natural mediante programación dinámica, que explota la estructura secuencial de las decisiones a lo largo del horizonte de planificación.

\begin{definition}[Formulación de Programación Dinámica]
Sea $f_t(I)$ la \textbf{función de valor óptimo} (función de Bellman) que representa el costo mínimo acumulado desde el periodo $t$ hasta el periodo $N$, dado un nivel de inventario inicial $I$ al comienzo del periodo $t$.

La ecuación de Bellman para este problema es:
\begin{equation}
f_t(I) = \min_{\substack{x_t \geq 0 \\ I + x_t \geq d_t}} \left[ C_t(I, x_t) + f_{t+1}(I + x_t - d_t) \right]
\tag{15}
\end{equation}
donde el \textbf{costo inmediato} $C_t(I, x_t)$ está dado por:
\begin{equation}
C_t(I, x_t) = h_{t-1} I + \delta(x_t) s_t
\tag{16}
\end{equation}

La condición terminal que cierra la recursión es:
\begin{equation}
f_{N+1}(I) = \begin{cases} 
0 & \text{si } I = 0 \\
\infty & \text{si } I > 0
\end{cases}
\quad \text{para todo } I \geq 0
\tag{17}
\end{equation}
\end{definition}

\begin{remark}[Interpretación de la Función de Bellman]
La función $f_t(I)$ encapsula el \textbf{principio de optimalidad} de Bellman aplicado a la gestión de inventarios:
\begin{itemize}
    \item \textbf{Estado del sistema}: El inventario inicial $I$ al comienzo del periodo $t$
    \item \textbf{Variable de control}: La cantidad a ordenar $x_t$
    \item \textbf{Transición de estado}: $I_{t+1} = I + x_t - d_t$
    \item \textbf{Costo inmediato}: $C_t(I, x_t)$ que incluye costos de mantenimiento y setup
    \item \textbf{Función de valor futuro}: $f_{t+1}(I + x_t - d_t)$ representa las consecuencias óptimas de la decisión actual
\end{itemize}
La ecuación (15) expresa que la política óptima en el estado $(t, I)$ consiste en elegir $x_t$ que minimice la suma del costo inmediato más el costo futuro óptimo.
\end{remark}

\begin{proposition}[Consistencia con el Principio de Optimalidad]
La formulación (15)-(17) satisface el principio de optimalidad de Bellman. Es decir, cualquier subpolítica de una política óptima es a su vez óptima para el subproblema correspondiente.
\end{proposition}

\begin{proof}
Sea $\pi^* = (x_1^*, x_2^*, \ldots, x_N^*)$ una política óptima para el problema completo, y sea $I_t^*$ la trayectoria de inventario resultante. Considere el subproblema desde el periodo $t$ hasta $N$ con inventario inicial $I_t^*$. 

Si existiera una política $\pi' = (x_t', \ldots, x_N')$ con costo menor que $(x_t^*, \ldots, x_N^*)$ para este subproblema, entonces la política $(x_1^*, \ldots, x_{t-1}^*, x_t', \ldots, x_N')$ tendría menor costo total que $\pi^*$, contradiciendo la optimalidad de $\pi^*$.
\end{proof}

\begin{remark}[Complejidad Computacional de la Implementación Directa]
Aunque la formulación (15)-(17) es conceptualmente correcta, su implementación computacional directa presenta serias dificultades:

\begin{itemize}
    \item \textbf{Espacio de estados continuo}: La variable de estado $I$ toma valores en $\mathbb{R}_{\geq 0}$, requiriendo discretización
    \item \textbf{Dimensionalidad}: Para cada periodo $t$, debemos computar $f_t(I)$ para múltiples valores de $I$
    \item \textbf{Complejidad exponencial}: Sin estructura adicional, el número de estados a considerar crece exponencialmente con $N$
    \item \textbf{Problemas de discretización}: La elección del paso de discretización afecta precisión y costo computacional
\end{itemize}

En el peor caso, si discretizamos el inventario en $K$ niveles, la complejidad sería $O(K^2 N)$, que puede ser prohibitiva para $K$ grande.
\end{remark}

\begin{example}[Ilustración de la Complejidad]
Para un problema con $N=12$ periodos y discretizando el inventario en $K=100$ niveles:
\begin{itemize}
    \item Número de estados: $12 \times 100 = 1,200$
    \item Evaluaciones de la ecuación de Bellman: $\approx 100^2 \times 12 = 120,000$
    \item Cálculo de mínimos sobre espacios continuos aproximados
\end{itemize}
Esto contrasta con la complejidad $O(N^2)$ del algoritmo Wagner-Whitin que explota las propiedades estructurales.
\end{example}

\begin{theorem}[Reducción del Espacio de Estados]
Gracias a la \textbf{propiedad de cero inventarios} (Teorema 1), podemos restringir la atención a estados donde $I = 0$ al comienzo de cada periodo. Esto reduce el espacio de estados de dimensión continua a uno discreto finito.
\end{theorem}

\begin{proof}
Por el Teorema 1, existe una solución óptima que satisface $I_{t-1} \cdot x_t = 0$ para todo $t$. Por lo tanto, en cada periodo $t$, o bien $I_{t-1} = 0$ (inventario cero) o bien $x_t = 0$ (no se ordena). 

En particular, al comienzo de cualquier periodo donde se realice un pedido, debemos tener $I = 0$. Como las decisiones de ordenar ocurren solo en puntos de regeneración, basta con considerar $f_t(0)$ para todo $t$.
\end{proof}

\begin{remark}[Simplificación Fundamental]
Los teoremas fundamentales de Wagner-Whitin permiten las siguientes simplificaciones cruciales:
\begin{enumerate}
    \item \textbf{Reducción del espacio de estados}: Solo necesitamos considerar $f_t(0)$
    \item \textbf{Patrón de pedidos}: Cada pedido cubre demandas consecutivas
    \item \textbf{Horizontes de planificación}: El problema se descompone en subproblemas independientes
\end{enumerate}
Estas propiedades reducen la complejidad de $O(K^2 N)$ a $O(N^2)$, haciendo el problema computacionalmente tratable incluso para $N$ grande.
\end{remark}

\begin{algorithm}[H]
\caption{Programación Dinámica Simplificada por Wagner-Whitin}
\begin{algorithmic}[1]
\State Inicializar $F(0) = 0$ \Comment{Costo óptimo hasta el periodo 0}
\For{$t = 1$ to $N$}
    \State $F(t) = \infty$
    \For{$j = 0$ to $t-1$}
        \State Calcular costo de ordenar en $j+1$ para cubrir periodos $j+1$ hasta $t$
        \State $costo = F(j) + s_{j+1} + \sum_{k=j+1}^{t-1} \sum_{l=k+1}^t h_l d_k$
        \If{$costo < F(t)$}
            \State $F(t) = costo$
            \State $P(t) = j$ \Comment{Registrar punto de regeneración}
        \EndIf
    \EndFor
\EndFor
\end{algorithmic}
\end{algorithm}

\begin{remark}[Equivalencia con la Formulación Original]
La formulación simplificada $F(t) = \min_{0 \leq j < t} [F(j) + s_{j+1} + H(j,t)]$ es equivalente a resolver la ecuación de Bellman (15) bajo las propiedades estructurales de Wagner-Whitin, donde $H(j,t)$ representa el costo de mantener inventario al cubrir demandas de $j+1$ a $t$ con una orden en $j+1$.
\end{remark}
\section{Resultados del articulo base}

Los siguientes teoremas, demostrados originalmente por Wagner y Whitin, establecen propiedades estructurales de la solución óptima que reducen drásticamente el espacio de búsqueda.

\begin{theorem}[Propiedad de Regeneración o Cero-Inventario]
Existe una solución óptima tal que para cada periodo $t$ se cumple $I_{t-1} \cdot x_t = 0$. Es decir, nunca se ordena y se tiene inventario positivo simultáneamente.
\end{theorem}

\begin{proof}
Supóngase una solución óptima donde para algún periodo $t$ se tiene $I_{t-1} > 0$ y $x_t > 0$. Consideremos una solución alternativa donde se reduce el pedido del periodo anterior en $\Delta > 0$ y se aumenta el pedido del periodo $t$ en la misma cantidad. El cambio en el costo sería:
\[
\Delta C = -h_{t-1} \Delta + \delta(x_{t-1} - \Delta) s_{t-1} - \delta(x_{t-1}) s_{t-1}
\]
Para $\Delta$ suficientemente pequeño, si $x_{t-1} - \Delta > 0$, entonces $\Delta C = -h_{t-1} \Delta < 0$, contradiciendo la optimalidad. Si $x_{t-1} - \Delta = 0$, entonces $\Delta C = -s_{t-1} - h_{t-1} \Delta < 0$, nuevamente una contradicción. Por lo tanto, debe cumplirse $I_{t-1} \cdot x_t = 0$.
\end{proof}

\begin{theorem}[Patrón de Pedidos o Satisfacción Exacta de Demanda]
Existe una solución óptima tal que cada pedido $x_t$ cubre exactamente la demanda de un número entero de periodos consecutivos. Formalmente, para cada $t$ con $x_t > 0$, existe $k \geq t$ tal que:
\begin{equation}
x_t = \sum_{j=t}^k d_j
\end{equation}
\end{theorem}

\begin{proof}
Directa del Teorema 1. Si un pedido no satisface exactamente la demanda de periodos consecutivos, existiría algún periodo intermedio con inventario positivo y donde se realiza un pedido, violando la propiedad de regeneración.
\end{proof}

\begin{theorem}[Horizonte de Planificación]
Si al resolver el problema para los primeros $t$ periodos se encuentra que el último pedido ocurre en el periodo $j^* \leq t$, entonces la solución óptima para el problema completo puede obtenerse concatenando la solución óptima para los primeros $j^*-1$ periodos (considerados independientemente) con la solución para los periodos restantes.
\end{theorem}

\begin{proof}
Sea $F(t)$ el costo mínimo para los primeros $t$ periodos. Por la propiedad de regeneración, existe un último periodo de pedido $j^* \leq t$ que satisface la demanda hasta $t$. El costo total sería $F(j^*-1) + s_{j^*} + \sum_{k=j^*}^{t-1} \sum_{l=k+1}^t h_l d_k$. Si este es el mínimo global para el subproblema de $t$ periodos, cualquier extensión a periodos futuros no modificará la optimalidad de esta decisión para los primeros $j^*-1$ periodos.
\end{proof}

\begin{corollary}[Formulación Recursiva Simplificada]
El costo mínimo para los primeros $t$ periodos puede expresarse como:
\begin{equation}
F(t) = \min_{0 \leq j \leq t} \left[ F(j) + s_{j+1} + \sum_{k=j+1}^{t-1} \sum_{l=k+1}^t h_l d_k \right]
\end{equation}
con $F(0) = 0$.
\end{corollary}

\section{Algoritmo Wagner-Whitin}

Los teoremas anteriores permiten formular un algoritmo forward que resuelve el problema de manera eficiente.

\subsection{Formulación Recursiva}

Sea $F(t)$ el costo mínimo para los primeros $t$ periodos, con $F(0) = 0$. Entonces:

\begin{equation}
F(t) = \min_{0 \leq j < t} \left[ F(j) + s_{j+1} + \sum_{k=j+1}^{t-1} \sum_{l=k+1}^t h_l d_k \right]
\end{equation}

donde el término $\sum_{k=j+1}^{t-1} \sum_{l=k+1}^t h_l d_k$ representa el costo de mantener el inventario necesario para satisfacer las demandas de los periodos $k = j+1, \ldots, t-1$ hasta sus respectivos periodos de consumo.

\subsection{Descripción del Algoritmo}

\begin{algorithm}
\caption{Algoritmo Wagner-Whitin}
\begin{algorithmic}[1]
\Require Demandas $d_t$, costos de setup $s_t$, costos de mantener $h_t$ para $t=1,\ldots,N$
\Ensure Política óptima de pedidos y costo total mínimo
\State Inicializar $F(0) = 0$, $P(0) = 0$ \Comment{$P(t)$ almacena el último periodo de pedido}
\For{$t = 1$ to $N$}
    \State $F(t) \gets \infty$
    \For{$j = 0$ to $t-1$}
        \State $costo \gets F(j) + s_{j+1} + \sum_{k=j+1}^{t-1} \sum_{l=k+1}^t h_l d_k$
        \If{$costo < F(t)$}
            \State $F(t) \gets costo$
            \State $P(t) \gets j$ \Comment{El próximo pedido cubre de $j+1$ a $t$}
        \EndIf
    \EndFor
\EndFor
\State Reconstruir política óptima mediante backtracking usando $P(t)$
\State \Return $F(N)$, política óptima
\end{algorithmic}
\end{algorithm}

\subsection{Complejidad Computacional y Optimizaciones}

El algoritmo básico requiere evaluar $O(N^2)$ posibles combinaciones $(j,t)$, con cada evaluación con costo $O(N)$, resultando en una complejidad total de $O(N^3)$. 

\begin{theorem}[Complejidad Mejorada]
Es posible implementar el algoritmo Wagner-Whitin con complejidad $O(N^2)$ mediante precomputación de costos de mantenimiento.
\end{theorem}

\begin{proof}
Definiendo $H(j,t) = \sum_{k=j}^{t-1} \sum_{l=k+1}^t h_l d_k$, podemos precomputar estos valores en $O(N^2)$ usando:
\[
H(j,t) = H(j,t-1) + d_{t-1} \sum_{l=t}^t h_l \quad \text{para } j < t
\]
con $H(j,j) = 0$.
\end{proof}

\section{Ejemplo}

Considérese el ejemplo original de Wagner-Whitin con $N = 12$ periodos, costos de setup $s_t$ variables, demandas $d_t$ variables, y costo de mantenimiento $h_t = 1$ para todo $t$. Los datos se presentan en la Tabla 1.

\begin{table}[h]
\centering
\caption{Datos del problema de 12 periodos}
\begin{tabular}{cccc}
\toprule
Mes $t$ & Demanda $d_t$ & Costo Setup $s_t$ \\
\midrule
1 & 69 & 85 \\
2 & 29 & 102 \\
3 & 36 & 102 \\
4 & 61 & 101 \\
5 & 61 & 98 \\
6 & 26 & 114 \\
7 & 34 & 105 \\
8 & 67 & 86 \\
9 & 45 & 119 \\
10 & 67 & 110 \\
11 & 79 & 98 \\
12 & 56 & 114 \\
\bottomrule
\end{tabular}
\end{table}

La aplicación del algoritmo produce la política óptima:
\begin{itemize}
    \item Ordenar en periodo 1: $x_1 = 69 + 29 = 98$
    \item Ordenar en periodo 3: $x_3 = 36 + 61 = 97$
    \item Ordenar en periodo 5: $x_5 = 61 + 26 + 34 = 121$
    \item Ordenar en periodo 8: $x_8 = 67 + 45 = 112$
    \item Ordenar en periodo 10: $x_{10} = 67$
    \item Ordenar en periodo 11: $x_{11} = 79 + 56 = 135$
\end{itemize}

El costo total óptimo es $F(12) = 864$.

\

El marco teórico establecido -basado en programación dinámica, optimización y teoría de inventarios- proporciona fundamentos sólidos para comprender tanto el algoritmo especí



\section*{Referencias}

\bibliographystyle{plain}
\begin{thebibliography}{9}

\bibitem{wagner1958dynamic}
Wagner, H. M. \& Whitin, T. M. (1958). Dynamic Version of the Economic Lot Size Model. \emph{Management Science}, 5(1), 89-96.

\bibitem{aksoy2024robust}
Aksoy, A. \& Küçükyavuz, S. (2024). Robust Lot-Sizing Problems with Uncertain Costs. \emph{Optimization Letters}, 18(3), 543-567.

\bibitem{bellman1957dynamic}
Bellman, R. (1957). \emph{Dynamic Programming}. Princeton University Press.

\bibitem{bravo2023stochastic}
Bravo, F. \& Vidal, C. J. (2023). Stochastic Lot-Sizing: Recent Advances and Future Directions. \emph{European Journal of Operational Research}, 305(2), 501-520.

\end{thebibliography}

\end{document}